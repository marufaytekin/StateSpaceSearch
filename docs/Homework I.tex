\documentclass{article}
\usepackage{amsmath}
\usepackage{amssymb}
\usepackage{algorithmic}
\usepackage[table]{xcolor}
\usepackage[linesnumbered,ruled,vlined]{algorithm2e}
\usepackage{listings}
\usepackage{color}
%\usepackage{algorithm}
\usepackage{caption}
\usepackage{subcaption}
\usepackage[numbers, sort]{natbib}
%\usepackage{floatrow}
\usepackage{graphicx}
\usepackage{verbatim} 
\usepackage{booktabs}
\usepackage{tabularx}
\usepackage{mathtools}
\usepackage{graphicx}
\usepackage{epstopdf}
\usepackage[export]{adjustbox}
\usepackage{url}
\usepackage{epsfig}
\usepackage{multirow}
\usepackage{pbox}

\pagenumbering{gobble}

\addtolength{\topmargin}{-.975in}

\usepackage{geometry}
 \geometry{
 a4paper,
 total={210mm,297mm},
 left=20mm,
 right=20mm,
 top=20mm,
 bottom=20mm,
 }
 
	
\title {Programming Assignment - I \\6160 Advanced Topics in Artificial Intelligence}

\author{\bf Maruf Aytekin \\
aaytekin@gmail.com}
s

\begin{document}

\maketitle

\section{Problem Definition}

Write a program to solve the 8-puzzle problem using the following search algorithms;
\begin{itemize}
  \item A* Search
  \item Breadth-first search
  \item Uniform-cost search
  \item Depth-first search
  \item Iterative deepening search
  \item Greedy Best Search
\end{itemize}

\section{Data Set}
8puzzles that is needs to be less than 15 used as inputs to calculate both the number of nodes expanded and the maximum number of nodes saved in the memory.

\section{Results}

{\setlength{\extrarowheight}{5pt}

\begin{table}[!ht]
\begin{tabularx}{\textwidth}{l  l  XXXXXXXXXX | l}
& & \multicolumn{10}{c} {Puzzle (1-10)} & Median \\
\toprule
\multirow{2}{*}{A* Search} & Expanded  & 4 & 3 & 2 & 3 & 4 & 5 & 14 & 9 & 8 & 9 &  6.1 \\
& In Memory & 9 & 9 & 9 & 9 & 9 & 11 & 23 & 23 & 23 & 23 &  14.8 \\
\hline
\multirow{2}{*}{\parbox{2.0cm}{Depth First Search}} & Expanded  & 36726& 15& 5965& 13708& 58475& 12076& 16963& 92145& 172918& 13956 & 42294.7 \\
& In Memory & 25& 25& 25& 25& 25& 25& 25& 28& 28& 28 & 25.9 \\
\hline
\multirow{2}{*}{\parbox{2.0cm}{Breadth First Search}} & Expanded  & 32 & 14 & 9 & 11 & 35 & 53 & 160 & 447 & 210 & 451 &  142.2 \\
& In Memory & 27 & 15 & 11 & 13 & 29 & 47 & 129 & 283 & 152 & 286 &  99.2 \\
\hline 
\multirow{2}{*}{\parbox{2.0cm}{Uniform Cost Search}} & Expanded  & 24 & 10 & 6 & 18 & 37 & 66 & 205 & 464 & 227 & 443 &  150.0 \\
& In Memory & 29 & 29 & 29 & 29 & 33 & 62 & 158 & 307 & 307 & 307 &  129.0 \\
\hline
\multirow{2}{*}{\parbox{2.0cm}{Greedy Best Search}} & Expanded  & 4 & 3 & 2 & 3 & 4 & 5 & 7 & 8 & 7 & 8 &  5.1 \\
& In Memory & 9 & 9 & 9 & 9 & 9 & 11 & 16 & 17 & 17 & 17 &  12.3\\
\hline
\multirow{2}{*}{\parbox{2.0cm}{Iterative Deepening Search}} & Expanded  & 36 & 25 & 13 & 40 & 20 & 11 & 13 & 450 & 2461 & 1879 &  494.8 \\
& In Memory & 8 & 8 & 8 & 8 & 8 & 9 & 15 & 15 & 16 & 16 &  11.1 \\
\bottomrule
\end{tabularx}
\end{table}

\section{Conclusion}

We limited to depth first search because DFS algorithm gets lost or causes out of memory exception most of the time. We limited depth first search by setting 15 as cut-off point.

In terms of expended node numbers A* search and greedy best search seems performed better when they are compared against the others. However, greedy best search gets lost in loops or unable to find a solution. The data set is chosen to make sure greedy search does not get lost.


\end{document}


